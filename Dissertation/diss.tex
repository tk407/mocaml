% The master copy of this demo dissertation is held on my filespace
% on the cl file serve (/homes/mr/teaching/demodissert/)

% Last updated by MR on 2 August 2001

\documentclass[12pt,twoside,notitlepage]{report}

\usepackage{a4}
\usepackage{verbatim}

\input{epsf}                            % to allow postscript inclusions
% On thor and CUS read top of file:
%     /opt/TeX/lib/texmf/tex/dvips/epsf.sty
% On CL machines read:
%     /usr/lib/tex/macros/dvips/epsf.tex



\raggedbottom                           % try to avoid widows and orphans
\sloppy
\clubpenalty1000%
\widowpenalty1000%

\addtolength{\oddsidemargin}{6mm}       % adjust margins
\addtolength{\evensidemargin}{-8mm}

\renewcommand{\baselinestretch}{1.1}    % adjust line spacing to make
                                        % more readable

\begin{document}

\bibliographystyle{plain}


%%%%%%%%%%%%%%%%%%%%%%%%%%%%%%%%%%%%%%%%%%%%%%%%%%%%%%%%%%%%%%%%%%%%%%%%
% Title


\pagestyle{empty}

\hfill{\LARGE \bf Tam\'as Kisp\'eter}

\vspace*{60mm}
\begin{center}
\Huge
{\bf Monadic Concurrency in OCaml} \\
\vspace*{5mm}
Part II in Computer Science \\
\vspace*{5mm}
Churchill College \\
\vspace*{5mm}
\today  % today's date
\end{center}

\cleardoublepage

%%%%%%%%%%%%%%%%%%%%%%%%%%%%%%%%%%%%%%%%%%%%%%%%%%%%%%%%%%%%%%%%%%%%%%%%%%%%%%
% Proforma, table of contents and list of figures

\setcounter{page}{1}
\pagenumbering{roman}
\pagestyle{plain}

\chapter*{Proforma}

{\large
\begin{tabular}{ll}
Name:               & \bf Tam\'as Kisp\'eter                     \\
College:            & \bf Churchill College                     \\
Project Title:      & \bf Monadic Concurrency in OCaml \\
Examination:        & \bf Part II in Computer Science, July 2014        \\
Word Count:         & \bf 1587\footnotemark[1]
(well less than the 12000 limit) \\
Project Originator: & Tam\'as Kisp\'eter                    \\
Supervisor:         & Jeremy Yallop                    \\ 
\end{tabular}
}
\footnotetext[1]{This word count was computed
by {\tt detex diss.tex | tr -cd '0-9A-Za-z $\tt\backslash$n' | wc -w}
}
\stepcounter{footnote}


\section*{Original Aims of the Project}

To write an OCaml framework for lightweight threading framework. This framework should be defined from basic semantics and have these semantics represented in a theorem prover setting for verification. The verification should include proofs of basic monadic laws. This theorem prover representation should be extracted to OCaml where the extracted code should be as faithful to the representation as possible. The extracted code should be able to run OCaml code concurrently.

To write a demonstration dissertation\footnote{A normal footnote without the
complication of being in a table.} using \LaTeX\ to save
student's time when writing their own dissertations. The dissertation
should illustrate how to use the more common \LaTeX\ constructs. It
should include pictures and diagrams to show how these can be
incorporated into the dissertation.  It should contain the entire
\LaTeX\ source of the dissertation and the Makefile.  It should
explain how to construct an MSDOS disk of the dissertation in
Postscript format that can be used by the book shop for printing, and,
finally, it should have the prescribed layout and format of a diploma
dissertation.


\section*{Work Completed}

All that has been completed appears in this dissertation.

\section*{Special Difficulties}

Learning how to incorporate encapulated postscript into a \LaTeX\
document on both CUS and Thor.
 
\newpage
\section*{Declaration}

I, Tam\'as Kisp\'eter of Churchill College, being a candidate for Part II of the Computer
Science Tripos, hereby declare
that this dissertation and the work described in it are my own work,
unaided except as may be specified below, and that the dissertation
does not contain material that has already been used to any substantial
extent for a comparable purpose.

\bigskip
\leftline{Signed [signature]}

\medskip
\leftline{Date [date]}

\cleardoublepage

\tableofcontents

\listoffigures

\newpage
\section*{Acknowledgements}

This document owes much to an earlier version written by Simon Moore
\cite{moore95}.  His help, encouragement and advice was greatly 
appreciated.

%%%%%%%%%%%%%%%%%%%%%%%%%%%%%%%%%%%%%%%%%%%%%%%%%%%%%%%%%%%%%%%%%%%%%%%
% now for the chapters

\cleardoublepage        % just to make sure before the page numbering
                        % is changed

\setcounter{page}{1}
\pagenumbering{arabic}
\pagestyle{headings}

\chapter{Introduction}

\section{Overview of the files}

This document consists of the following files:

\begin{itemize}
\item {\tt Makefile} --- The Makefile for the dissertation and Project Proposal
\item {\tt diss.tex} --- The dissertation
\item {\tt propbody.tex} --- Appendix~C  -- the project proposal 
\item {\tt proposal.tex}  --- A \LaTeX\ main file for the proposal 
\item{\tt figs} -- A directory containing diagrams and pictures
\item{\tt refs.bib} --- The bibliography database
\end{itemize}

\section{Building the document}

This document was produced using \LaTeXe which is based upon
\LaTeX\cite{Lamport86}.  To build the document you first need to
generate {\tt diss.aux} which, amongst other things, contains the
references used.  This if done by executing the command:

{\tt latex diss}

\noindent
Then the bibliography can be generated from {\tt refs.bib} using:

{\tt bibtex diss}

\noindent
Finally, to ensure all the page numbering is correct run {\tt latex}
on {\tt diss.tex} until the {\tt .aux} files do not change.  This
usually takes 2 more runs.

\subsection{The makefile}

To simplify the calls to {\tt latex} and {\tt bibtex}, 
a makefile has been provided, see Appendix~\ref{makefile}. 
It provides the following facilities:

\begin{itemize}

\item{\tt make} \\
 Display help information.

\item{\tt make prop} \\
 Run {\tt latex proposal; xdvi proposal.dvi}.

\item{\tt make diss.ps} \\
 Make the file {\tt diss.ps}.

\item{\tt make gv} \\
 View the dissertation using ghostview after performing 
{\tt make diss.ps}, if necessary.

\item{\tt make gs} \\
 View the dissertation using ghostscript after performing 
{\tt make diss.ps}, if necessary.

\item{\tt make count} \\
Display an estimate of the word count.

\item{\tt make all} \\
Construct {\tt proposal.dvi} and {\tt diss.ps}.

\item{\tt make pub} \\ Make a {\tt .tar} version of the {\tt demodiss}
directory and place it in my {\tt public\_html} directory.

\item{\tt make clean} \\ Delete all files except the source files of
the dissertation. All these deleted files can be reconstructed by
typing {\tt make all}.

\item{\tt make pr} \\
Print the dissertation on your default printer.

\end{itemize}


\section{Counting words}

An approximate word count of the body of the dissertation may be
obtained using:

{\tt wc diss.tex}

\noindent
Alternatively, try something like:

\verb/detex diss.tex | tr -cd '0-9A-Z a-z\n' | wc -w/




\cleardoublepage



\chapter{Preparation}

This chapter is empty!


\cleardoublepage
\chapter{Implementation}

\section{Verbatim text}

Verbatim text can be included using \verb|\begin{verbatim}| and
\verb|\end{verbatim}|. I normally use a slightly smaller font and
often squeeze the lines a little closer together, as in:

{\renewcommand{\baselinestretch}{0.8}\small\begin{verbatim}
GET "libhdr"
 
GLOBAL { count:200; all  }
 
LET try(ld, row, rd) BE TEST row=all
                        THEN count := count + 1
                        ELSE { LET poss = all & ~(ld | row | rd)
                               UNTIL poss=0 DO
                               { LET p = poss & -poss
                                 poss := poss - p
                                 try(ld+p << 1, row+p, rd+p >> 1)
                               }
                             }
LET start() = VALOF
{ all := 1
  FOR i = 1 TO 12 DO
  { count := 0
    try(0, 0, 0)
    writef("Number of solutions to %i2-queens is %i5*n", i, count)
    all := 2*all + 1
  }
  RESULTIS 0
}
\end{verbatim}
}

\section{Tables}

\begin{samepage}
Here is a simple example\footnote{A footnote} of a table.

\begin{center}
\begin{tabular}{l|c|r}
Left      & Centred & Right \\
Justified &         & Justified \\[3mm]
%\hline\\%[-2mm]
First     & A       & XXX \\
Second    & AA      & XX  \\
Last      & AAA     & X   \\
\end{tabular}
\end{center}

\noindent
There is another example table in the proforma.
\end{samepage}

\section{Simple diagrams}

Simple diagrams can be written directly in \LaTeX.  For example, see
figure~\ref{latexpic1} on page~\pageref{latexpic1} and see
figure~\ref{latexpic2} on page~\pageref{latexpic2}.

\begin{figure}
\setlength{\unitlength}{1mm}
\begin{center}
\begin{picture}(125,100)
\put(0,80){\framebox(50,10){AAA}}
\put(0,60){\framebox(50,10){BBB}}
\put(0,40){\framebox(50,10){CCC}}
\put(0,20){\framebox(50,10){DDD}}
\put(0,00){\framebox(50,10){EEE}}

\put(75,80){\framebox(50,10){XXX}}
\put(75,60){\framebox(50,10){YYY}}
\put(75,40){\framebox(50,10){ZZZ}}

\put(25,80){\vector(0,-1){10}}
\put(25,60){\vector(0,-1){10}}
\put(25,50){\vector(0,1){10}}
\put(25,40){\vector(0,-1){10}}
\put(25,20){\vector(0,-1){10}}

\put(100,80){\vector(0,-1){10}}
\put(100,70){\vector(0,1){10}}
\put(100,60){\vector(0,-1){10}}
\put(100,50){\vector(0,1){10}}

\put(50,65){\vector(1,0){25}}
\put(75,65){\vector(-1,0){25}}
\end{picture}
\end{center}
\caption{\label{latexpic1}A picture composed of boxes and vectors.}
\end{figure}

\begin{figure}
\setlength{\unitlength}{1mm}
\begin{center}

\begin{picture}(100,70)
\put(47,65){\circle{10}}
\put(45,64){abc}

\put(37,45){\circle{10}}
\put(37,51){\line(1,1){7}}
\put(35,44){def}

\put(57,25){\circle{10}}
\put(57,31){\line(-1,3){9}}
\put(57,31){\line(-3,2){15}}
\put(55,24){ghi}

\put(32,0){\framebox(10,10){A}}
\put(52,0){\framebox(10,10){B}}
\put(37,12){\line(0,1){26}}
\put(37,12){\line(2,1){15}}
\put(57,12){\line(0,2){6}}
\end{picture}

\end{center}
\caption{\label{latexpic2}A diagram composed of circles, lines and boxes.}
\end{figure}



\section{Adding more complicated graphics}

The use of \LaTeX\ format can be tedious and it is often better to use
encapsulated postscript to represent complicated graphics.
Figure~\ref{epsfig} and ~\ref{xfig} on page \pageref{xfig} are
examples. The second figure was drawn using {\tt xfig} and exported in
{\tt.eps} format. This is my recommended way of drawing all diagrams.


\begin{figure}[tbh]
%\centerline{\epsfbox{figs/cuarms.eps}}
\caption{\label{epsfig}Example figure using encapsulated postscript}
\end{figure}

\begin{figure}[tbh]
\vspace{4in}
\caption{\label{pastedfig}Example figure where a picture can be pasted in}
\end{figure}


\begin{figure}[tbh]
%\centerline{\epsfbox{figs/diagram.eps}}
\caption{\label{xfig}Example diagram drawn using {\tt xfig}}
\end{figure}




\cleardoublepage
\chapter{Evaluation}

\section{Printing and binding}

If you have access to a laser printer that can print on two sides, you
can use it to print two copies of your dissertation and then get them
bound by the Computer Laboratory Bookshop. Otherwise, print your
dissertation single sided and get the Bookshop to copy and bind it double
sided.


Better printing quality can sometimes be obtained by giving the
Bookshop an MSDOS 1.44~Mbyte 3.5" floppy disc containing the
Postscript form of your dissertation. If the file is too large a
compressed version with {\tt zip} but not {\tt gnuzip} nor {\tt
compress} is acceptable. However they prefer the uncompressed form if
possible. From my experience I do not recommend this method.

\subsection{Things to note}

\begin{itemize}
\item Ensure that there are the correct number of blank pages inserted
so that each double sided page has a front and a back.  So, for
example, the title page must be followed by an absolutely blank page
(not even a page number).

\item Submitted postscript introduces more potential problems.
Therefore you must either allow two iterations of the binding process
(once in a digital form, falling back to a second, paper, submission if
necessary) or submit both paper and electronic versions.

\item There may be unexpected problems with fonts.

\end{itemize}

\section{Further information}

See the Computer Lab's world wide web pages at URL:

{\tt http://www.cl.cam.ac.uk/TeXdoc/TeXdocs.html}


\cleardoublepage
\chapter{Conclusion}

I hope that this rough guide to writing a dissertation is \LaTeX\ has
been helpful and saved you time.




\cleardoublepage

%%%%%%%%%%%%%%%%%%%%%%%%%%%%%%%%%%%%%%%%%%%%%%%%%%%%%%%%%%%%%%%%%%%%%
% the bibliography

\addcontentsline{toc}{chapter}{Bibliography}
\bibliography{refs}
\cleardoublepage

%%%%%%%%%%%%%%%%%%%%%%%%%%%%%%%%%%%%%%%%%%%%%%%%%%%%%%%%%%%%%%%%%%%%%
% the appendices
\appendix

\chapter{Latex source}

\section{diss.tex}
{\scriptsize\verbatiminput{diss.tex}}

\section{proposal.tex}
{\scriptsize\verbatiminput{proposal.tex}}

\section{propbody.tex}
{\scriptsize\verbatiminput{propbody.tex}}



\cleardoublepage

\chapter{Makefile}

\section{\label{makefile}Makefile}
{\scriptsize\verbatiminput{makefile.txt}}

\section{refs.bib}
{\scriptsize\verbatiminput{refs.bib}}


\cleardoublepage

\chapter{Project Proposal}



\section{Introduction of work to be undertaken}
With the rise of ubiquitous multiple core systems it is necessary for a working programmer to use concurrency to the greatest extent. However concurrent code has never been easy to write as human reasoning is often poorly equipped with the tools necessary to think about such systems. That is why it is essential for a programming language to provide safe and sound primitives to tackle this problem. 

My project aims to do this in the OCaml\cite{OCaml} language by developing a lightweight cooperating threading framework that holds correctness as a core value. The functional nature allows the use of one of the most recent trends in languages popular in academia, monads, to be used for a correct implementation. 

There have been two very successful frameworks, LWT\cite{LWT} and Async\cite{Async} that both provided the primitives for concurrent development in OCaml however neither is supported by a clear semantic description as their main focus was ease of use and speed. 
\section{Description of starting point}
My personal starting points are the courses ML under Windows (IA), Semantics of Programming Languages (IB), Logic and Proof (IB) and Concepts in Programming Languages (IB). Furthermore I have done extracurricular reading into semantics and typing and attended the Denotational Semantics (II) course in the past year.

The preparatory research period has to include familiarising myself with OCaml and the chosen specification and proof assistant tools.
\section{Substance and structure}
The project will consist of first creating a formal specification for a simple monad that has three main operations bind, return and choose. The behaviour of these operations will be specified in a current semantics tool like Lem\cite{Lem} or Ott\cite{Ott}. 

As large amount of research has gone into both monadic concurrency and implementations in OCaml, the project will draw inspiration from Claessen\cite{Claessen99functionalpearls}, Deleuze\cite{deleuzelight} and Vouillon\cite{vouillon2008lwt}.

Some atomic, blocking operations will also be specified including reading and writing to a console prompt or file to better illustrate the concurrency properties and make testing and evaluation possible.

This theory driven executable specification will be paired by a hand implementation and will be thoroughly checked against each other to ensure that both adhere to the desired semantics. 

Both of these implementations will be then compared against the two current frameworks for simplicity and speed on various test cases.

If time allows, an extension will also be carried out on the theorem prover version of the specification to formally verify that the implementation is correct. 
\section{Criteria}
For the project to be deemed a success the following items must be
successfully completed.
\begin{enumerate}
\item A specification for a monadic concurrency framework must be designed in the format of a semantics tool.
\item This specification needs to be exported to a proof assistant and has a runnable OCaml version
\item Test cases must be written that can thoroughly check a concurrency framework
\item A hand implementation needs to be designed, implemented and tested against the specification
\item The implementations must be compared to the frameworks LWT and Async based on speed
\item The dissertation must be planned and written
\end{enumerate}

In case the extension will also become viable then its success criterion is that there is a clear formal verification accompanying the automated theorem prover version of the specification.
\section{Timetable}
The project will be split into two week packages


\subsection{Week 1 and 2}
Preparatory reading and research into tools that can be used for writing the specification and in the extension, the proofs. The tools of choice at the time of proposal are Ott for the specification step and Coq\cite{Coq} as the proof assistant. Potentially a meeting arranged in the Computer Lab by an expert in using these tools. 

\textbf{Deliverable:} Small example specifications to try out the tool chain, including SKI combinator calculus.

\subsection{Week 3 and 4}
Investigating the two current libraries and their design decisions and planning the necessary parts of specification. Identifying the test cases that are thorough and common in concurrent code.

\textbf{Deliverable:} A document describing the major design decisions of the two libraries, the difference in design of the specification and a set of test cases much like the ones used in OCaml Light \cite{OCamlLight, OCamlLightWeb}, but with a concurrency focus.

\subsection{Week 5 and 6}
Writing the specification and exporting to automated theorem provers and OCaml.

\textbf{Deliverable:} The specification document in the format of the semantics tool and exported in the formats of the proof assistant and OCaml.


\subsection{Week 7 and 8}
Hand implement a version that adheres to the specification and test it against the runnable semantics.

\subsection{Week 9 and 10}
Evaluating the implementations of the concurrency framework against LWT and Async.
Writing up the halfway report.

\textbf{Deliverable:} Evaluation data and charts, the halfway report.
\subsection{Week 11 and 12}
If unexpected complexity occurs these two weeks can be used to compensate, otherwise starting on the verification proof in the proof assistant.
\subsection{Week 13 and 14}
If necessary adding more primitives (I/O, network) to test with, improving performance and finishing the verification proof. If time allows writing guide for future use of the framework.
\subsection{Week 15 and 16}
Combining all previously delivered documents as a starting point for the dissertation and doing any necessary further evaluation and extension.
Creating the first, rough draft of the dissertation.
\subsection{Week 17 and 18}
Getting to the final structure but not necessarily final wording of the dissertation, acquiring all necessary graphs and charts, incorporating ongoing feedback from the supervisor.
\subsection{Week 19 and 20}
Finalising the dissertation and incorporating all feedback and polishing. 
\section{Resource Declaration}
The project will need the following resources:
\begin{itemize}
\item MCS computer access that is provided for all projects
\item The OCaml core libraries and compiler
\item The LWT and Async libraries
\item The Lem tool
\item The Ott tool
\item The use of my personal laptop, to work more efficiently
\end{itemize}

As my personal laptop is included a suitable back-up plan is necessary which will consist of the following:
\begin{itemize}
\item A backup to my personal Dropbox account
\item A Git repository on Github
\item Frequent backups (potentially remotely) to the MCS partition
\end{itemize}
My supervisor and on request my overseers will receive access to both the Dropbox account and Github repository to allow full transparency.


\begin{thebibliography}{9}

\bibitem{OCaml}
 OCaml.
 \emph{http://ocaml.org/}

\bibitem{LWT}
 LWT, Lightweight Threading library.
 \emph{http://ocsigen.org/lwt/}

\bibitem{Async}
 Async, Open source concurrency library by Jane Street
 \emph{http://janestreet.github.io/}
 
 \bibitem{Lem}
 Lem, a tool for lightweight executable mathematics.\newline
 \emph{http://www.cs.kent.ac.uk/people/staff/sao/lem/}
 
 \bibitem{Ott}
  Ott, a tool for writing definitions of programming languages and calculi.
  Francesco Zappa Nardelli, Peter Sewell, and Scott Owens.\newline
 \emph{http://www.cl.cam.ac.uk/\textasciitilde pes20/ott/}

\bibitem{Claessen99functionalpearls}
  Claessen, Koen.
  \emph{Functional Pearls: A Poor Man's Concurrency Monad}.
  1999.

\bibitem{deleuzelight}
  Deleuze, Christophe.
  \emph{Light Weight Concurrency in OCaml: Continuations, Monads, Promises, Events}.

\bibitem{vouillon2008lwt}
  Vouillon, J{\'e}r{\^o}me.
  \emph{Lwt: a cooperative thread library} in Proceedings of the 2008 ACM SIGPLAN workshop on ML, pages 3--12.
  2008.
  ACM.
 
\bibitem{Coq}
 Coq proof assistant.
 \emph{http://coq.inria.fr/}

\bibitem{OCamlLight}  
 Owens, Scott.
 \emph{A sound semantics for OCaml light.} in Programming Languages and Systems, pages 1--15.
 2008.
 Springer Berlin Heidelberg.
 
\bibitem{OCamlLightWeb}
 Owens, Scott,
 \emph{http://www.cl.cam.ac.uk/\textasciitilde so294/ocaml/}.
 2008.
 


\end{thebibliography}


\end{document}
